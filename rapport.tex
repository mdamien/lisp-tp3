\documentclass[a4paper, 12pt, leqno]{report}

\usepackage[T1]{fontenc}
\usepackage[utf8x]{inputenc}
\usepackage[greek,french]{babel}
\usepackage[babel=true]{csquotes}
\usepackage[top=1.5cm, bottom=2.0cm, left=2.0cm, right=2.0cm]{geometry}
\usepackage{fancyhdr}
\usepackage{graphicx}
\usepackage{amsmath}
\usepackage{soul}
\usepackage{amssymb}
\pagestyle{plain}
\usepackage{amsfonts}
\usepackage{amssymb}
\usepackage{amsthm}
\usepackage{graphicx}
\usepackage{color}
\usepackage{enumerate}
\usepackage{array}
\theoremstyle{plain}
\usepackage{blkarray}
\usepackage{listings}
\usepackage{algorithmicx}
\usepackage{algpseudocode}
\usepackage{algorithm}

\usepackage{caption}
\DeclareCaptionFont{white}{\color{white}}
\DeclareCaptionFormat{listing}{\colorbox{black}{\parbox{\textwidth}{#1#2#3}}}
\captionsetup[lstlisting]{format=listing,labelfont=white,textfont=white}

\title{\textsc{\textbf{Rapport TP n°3 IA01 \\ Diagnostic de panne de site}}}
\author{\textbf{Damien MARI\'E et Antoine POUILLAUDE}}

\begin{document}
    \maketitle
    \tableofcontents
\newpage
\section*{Remarques générales}
		Ce TP, bien que difficile, nous à ammener à implémenter des fonctions avancées en LISP et des moteurs généralistes de résolution de systéme expert 0+. Le plus difficile ayant été de trouver le sujet sur lequel nous avons appliquer notre systéme expert.

\chapter{Formalisation}  

Afin de développer un systéme expert d'odre 0+, il nous faut pour celà une formalisation de la base de faits, de la base de régles et de la méthode d'inférence utilisée pour le parcours en largeur comme en profondeur.

 \section{Base de régles}

Chaque régle est représentée par: Un identifiant, des premisses et des actions.

\begin{lstlisting}
( id
	( ( premisse1) (premisse2) ...)
	( ( action1) (action2) ...)
)
\end{lstlisting}

Chaque prémisse étant constitué d'un opérateur de comparaison et de deux valeurs:

\begin{lstlisting}
( operateur valeur1 valeur2 valeur3 ...)
\end{lstlisting}

Exemple d'opérateur: =, <, >

De même les actions sont définis par un opérateur et des valeurs:
\begin{lstlisting}
( operateur valeur1 valeur2 valeur3 ...)
\end{lstlisting}

Exemple d'opérateur: set, print-message, ask

 \section{Base de faits}

De façon beaucoup plus simple, chaque fait est représenté par

<<voir code d'Antoine -- a faire>>

 \section{Fonctionnement du moteur}

L'inférence peut être faite en profondeur ou en largeur, en voici les algorithme simplifés


\chapter{Application: Diagnostic de panne de site}
 \section{Choix du systéme expert}

Pour ce TP, nous avons décidé d'implémenter un systéme expert capable de trouver la cause et la solution à une panne informatique. Celà

 \chapter{Comparaison des moteurs}
                
\end{document}