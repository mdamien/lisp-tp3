\documentclass[a4paper, 12pt, leqno]{report}

\usepackage[T1]{fontenc}
\usepackage[utf8x]{inputenc}
\usepackage[greek,french]{babel}
\usepackage[babel=true]{csquotes}
\usepackage[top=1.5cm, bottom=2.0cm, left=2.0cm, right=2.0cm]{geometry}
\usepackage{fancyhdr}
\usepackage{graphicx}
\usepackage{amsmath}
\usepackage{soul}
\usepackage{amssymb}
\pagestyle{plain}
\usepackage{amsfonts}
\usepackage{amssymb}
\usepackage{amsthm}
\usepackage{graphicx}
\usepackage{color}
\usepackage{enumerate}
\usepackage{array}
\theoremstyle{plain}
\usepackage{blkarray}
\usepackage{listings}
\usepackage{algorithmicx}
\usepackage{algpseudocode}
\usepackage{algorithm}

\usepackage{caption}
\DeclareCaptionFont{white}{\color{white}}
\DeclareCaptionFormat{listing}{\colorbox{black}{\parbox{\textwidth}{#1#2#3}}}
\captionsetup[lstlisting]{format=listing,labelfont=white,textfont=white}

\title{\textsc{\textbf{Rapport TP n°3 IA01}}}
\author{\textbf{Damien MARI\'E et Antoine POUILLAUDE}}

\begin{document}
    \maketitle
    \tableofcontents
\newpage
\section*{Remarques générales}
		Ce TP, bien que difficile, nous à ammener à implémenter des fonctions avancées en LISP et des moteurs généralistes de résolution de systéme expert 0+. Le plus difficile ayant été de trouver le sujet sur lequel nous avons appliquer notre systéme expert.

\input{"chapter1.tex"}
\input{"chapter2.tex"}
\input{"chapter3.tex"}
                
\end{document}