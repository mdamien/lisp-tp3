\chapter{Formalisation}  

Afin de développer un systéme expert d'odre 0+, il nous faut pour celà une formalisation de la base de faits, de la base de régles et de la méthode d'inférence utilisée pour le parcours en largeur comme en profondeur.

 \section{Base de régles}

Chaque régle est représentée par: Un identifiant, des premisses et des actions.

\begin{lstlisting}
( id
	( ( premisse1) (premisse2) ...)
	( ( action1) (action2) ...)
)
\end{lstlisting}

Chaque prémisse étant constitué d'un opérateur de comparaison et de deux valeurs:

\begin{lstlisting}
( operateur valeur1 valeur2 valeur3 ...)
\end{lstlisting}

Exemple d'opérateur: =, <, >

De même les actions sont définis par un opérateur et des valeurs:
\begin{lstlisting}
( operateur valeur1 valeur2 valeur3 ...)
\end{lstlisting}

Exemple d'opérateur: set, print-message, ask

 \section{Base de faits}

De façon beaucoup plus simple, chaque fait est représenté par

<<voir code d'Antoine -- a faire>>